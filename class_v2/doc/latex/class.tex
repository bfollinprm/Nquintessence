\documentclass{article}

\begin{document}

\title{The Cosmic Linear Anisotropy Solving System (CLASS)}

\author{Julien Lesgourgues}

%\email{Julien.Lesgourgues@cern.ch}

\date{\today}

%\affiliation{CERN}

\maketitle

\section{Overall architecture of {\sc class}}

\subsection{The seven-module backbone}

The purpose of {\sc class} consists in computing some power spectra for a
given set of cosmological parameters. This task can be decomposed in few steps
or modules:
\begin{enumerate}
\item
compute the evolution of cosmological background quantitites.
\item
compute the evolution of thermodynamical quantitites (ionization fractions, etc.)
\item
compute the evolution of source functions $S(k,\eta)$ (by integrating over all perturbations).
\item
compute Bessel functions (in order to go from Fourier to harmonic space).
\item
compute transfer functions $\Delta_l(k)$ (unless one needs only Fourier spectra $P(k)$'s and no harmonic spectra $C_l$'s).
\item
compute the primordial spectrum for scalars, tensors, etc.
(straightforward if the input consists in spectral parameters $A_s$,
$n_s$, $r$, ..., but this module will incorporate the option of integrating
over inflationary perturbations).
\item
compute power spectra $C_l$'s and/or $P(k)$'s.
\end{enumerate}
In {\sc class}, each of these steps is associated with a structure:
\begin{enumerate}
\item {\tt  struct background }      for cosmological background,
\item {\tt  struct thermo      }     for thermodynamics,
\item {\tt  struct perturbs     }    for source functions,
\item {\tt  struct bessels  }        for bessel functions,
\item {\tt  struct transfers }       for transfer functions,
\item {\tt  struct primordial }      for primordial spectra,
\item {\tt  struct spectra     }     for output spectra. 
\end{enumerate}
A given structure contains ``everything concerning one step that the
subsequent steps need to know'' (for instance, everything about source
functions that the transfer module needs to know). In particular, each
structure contains one array of tabulated values (background
quantitites as a function of time, thermodynamical quantitites as a
function of redshift, sources as a function of $(k, \eta)$, etc.).  It
also contains information about the size of this array and the value
of the index of each physical quantity, so that the table can be
easily read and interpolated. Finally, it contains any derived
quantity that other modules might need to know. Hence, the
comunication from one module A to another module B
consists in passing a pointer to the structure filled by A, and nothing else.

Each structure is defined and filled in one of the following modules
(and precisely in the order below):
\begin{enumerate}
\item {\tt  background.c }
\item {\tt  thermodynamics.c   }
\item {\tt  perturbations.c    }
\item {\tt  bessel.c  } 
\item {\tt  transfer.c }
\item {\tt  primordial.c }
\item {\tt  spectra.c     }
\end{enumerate}
Each of these modules contains at least three functions:
\begin{itemize}
\item {\it module}\_{\tt init(...)}
\item {\it module}\_{\tt free()}
\item {\it module}\_{\it something}\_{\tt at}\_{\it somevalue}{\tt(...)}
\end{itemize}
The first function allocates and fills each structure. This can be
done provided that the previous structures in the hierarchy have been
already allocated and filled. In summary, calling one of {\it module}\_{\tt
init(...)} amounts in solving entirely one of the steps 1 to 7.\\ 
The second function deallocates the fields of each structure. This
can be done optionally at the end of the code (or, when the code is embedded
in a sampler, this {\it must} be done 
between each execution of {\sc class}, and especially before calling {\it
module}\_{\tt init(...)} again with different input parameters).\\ The
third function is able to interpolate the pre-computed tables. For
instance, {\tt background\_init()} fills a table of background
quantitites for discrete values of conformal time $\eta$, but {\tt
background\_at\_eta(eta, * values)} will return these values for any
arbitrary $\eta$. \\ Note that functions of the type {\it module}\_{\it
something}\_{\tt at}\_{\it somevalue}{\tt(...)} are the only ones which are
called from another module, while functions of the type {\it
module}\_{\tt init(...)} and {\it module}\_{\tt free()} are the only
one called by the main executable.  All other functions are for
internal use in each module.\\

\subsection{Input}

There are two types of input:
\begin{enumerate}
\item ``precision parameters'' (controlling the precision of the output and the execution time),
\item ``input parameters'' (cosmological parameters, flags telling to the code what it should compute, ...)
\end{enumerate}
All ``precision parameters'' have been grouped in a single structure
{\tt struct precision}. The code contains {\it no other arbitrary
numerical coefficient}. This structure is initialized in a simple
module {\tt precision.c} by the function {\tt
precision\_init()}. Nothing is allocated dynamically in this function,
so there is no need for a {\tt precision\_free()} function.\\ Each
``input parameter'' refers to one particular step in the computation:
background, thermodynamics, perturbations, etc. Hence they are defined
as part of the corresponding structure. Their values are assigned in a
simple module {\tt input.c}, by a function {\tt input\_init(...)}
which has a pointer towards each structure in its list of
arguments. Hence, when a given function {\it module}\_{\tt init(...)}
is called, the corresponding structure already contains input
parameters; the function fills the rest of this structure. The
function {\tt input\_init(...)} does not allocate any field
dynamically, so there is no need for an {\tt input\_free()} function.

\subsection{Output}

A simple module {\tt output.c} writes the final results in files. The
name of the files are considered as input parameters making part of a
small structure {\tt struct output}.  Like for all other input
parameters, these names are assigned inside the function {\tt
input\_init(...)}. Again this structure contains no dynamically allocated
quantitites, so there is no need for an {\tt output\_free()} function.

\subsection{Summary}

We hope that after this short overview, it is clear for the reader
that the main executable of {\sc class} should consist only in the 
following lines
(not including comments and error-management lines):

\vspace{0.5cm}

\noindent
{\tt
main() \{ \\
\indent  struct precision pr;\\
\indent  struct background ba;\\
\indent  struct thermo th;     \\
\indent  struct perturbs pt;   \\
\indent  struct bessels bs;    \\
\indent  struct transfers tr;  \\
\indent  struct primordial pm;  \\
\indent  struct spectra sp;     \\
\indent  struct output op;      \\
 \\
\indent  precision\_init(\&pr)\\
\indent  input\_init(\&ba,\&th,\&pt,\&bs,\&tr,\&pm,\&sp,\&op)\\
\indent  background\_init(\&pr,\&ba)\\
\indent  thermodynamics\_init(\&ba,\&pr,\&th)\\
\indent  perturb\_init(\&ba,\&th,\&pr,\&pt)\\
\indent  bessel\_init(\&ba,\&pt,\&pr,\&bs)\\
\indent  transfer\_init(\&ba,\&th,\&pt,\&bs,\&pr,\&tr)\\
\indent  primordial\_init(\&pt,\&pr,\&pm)\\
\indent  spectra\_init(\&pt,\&tr,\&pm,\&sp)\\
\indent  output\_init(\&pt,\&tr,\&sp,\&op)\\
\\
\indent  /****** done ******/\\
\\
\indent  spectra\_free()\\
\indent  primordial\_free()\\
\indent  transfer\_free()\\
\indent  bessel\_free()\\
\indent  perturb\_free()\\
\indent  thermodynamics\_free()\\
\indent background\_free()\\
\}
}

\vspace{0.5cm}

For a given purpose, somebody could only be interested in the
intermediate steps (only background quantitites, only the
thermodynamics, only the perturbations and sources, etc.) It is
then straightforward to truncate the full hierarchy of modules 1, ... 7 at some
arbitrary order. We provide several ``reduced executables'' 
{\tt test}\_{\it module} achieving precisely this.

Note also that if {\sc class} is embedded in a
parameter sampler and only ``fast'' parameters are varied (i.e.,
parameters related to the primordial spectra), then it is only
necessary to repeat the following steps after {\tt output\_init(...)}:

\vspace{0.5cm}

{\tt 
\indent  spectra\_free()\\
\indent  primordial\_free()\\
\indent  input\_init(\&ba,\&th,\&pt,\&bs,\&tr,\&pm,\&sp,\&op)\\
\indent  primordial\_init(\&pt,\&pr,\&pm)\\
\indent  spectra\_init(\&pt,\&tr,\&pm,\&sp)\\
\indent  output\_init(\&pt,\&tr,\&sp,\&op)\\
}

\section{General principles}

\subsection{Flexibility}

Explain allocation of indices, ...

\subsection{Control of precision}

Explain precision structure, ...

\subsection{Control of errors}

\end{document}
